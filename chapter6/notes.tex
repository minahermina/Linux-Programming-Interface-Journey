% Options for packages loaded elsewhere
\PassOptionsToPackage{unicode}{hyperref}
\PassOptionsToPackage{hyphens}{url}
%
\documentclass[
]{article}
\usepackage{amsmath,amssymb}
\usepackage{lmodern}
\usepackage{iftex}
\ifPDFTeX
  \usepackage[T1]{fontenc}
  \usepackage[utf8]{inputenc}
  \usepackage{textcomp} % provide euro and other symbols
\else % if luatex or xetex
  \usepackage{unicode-math}
  \defaultfontfeatures{Scale=MatchLowercase}
  \defaultfontfeatures[\rmfamily]{Ligatures=TeX,Scale=1}
\fi
% Use upquote if available, for straight quotes in verbatim environments
\IfFileExists{upquote.sty}{\usepackage{upquote}}{}
\IfFileExists{microtype.sty}{% use microtype if available
  \usepackage[]{microtype}
  \UseMicrotypeSet[protrusion]{basicmath} % disable protrusion for tt fonts
}{}
\makeatletter
\@ifundefined{KOMAClassName}{% if non-KOMA class
  \IfFileExists{parskip.sty}{%
    \usepackage{parskip}
  }{% else
    \setlength{\parindent}{0pt}
    \setlength{\parskip}{6pt plus 2pt minus 1pt}}
}{% if KOMA class
  \KOMAoptions{parskip=half}}
\makeatother
\usepackage{xcolor}
\usepackage{color}
\usepackage{fancyvrb}
\newcommand{\VerbBar}{|}
\newcommand{\VERB}{\Verb[commandchars=\\\{\}]}
\DefineVerbatimEnvironment{Highlighting}{Verbatim}{commandchars=\\\{\}}
% Add ',fontsize=\small' for more characters per line
\newenvironment{Shaded}{}{}
\newcommand{\AlertTok}[1]{\textcolor[rgb]{1.00,0.00,0.00}{\textbf{#1}}}
\newcommand{\AnnotationTok}[1]{\textcolor[rgb]{0.38,0.63,0.69}{\textbf{\textit{#1}}}}
\newcommand{\AttributeTok}[1]{\textcolor[rgb]{0.49,0.56,0.16}{#1}}
\newcommand{\BaseNTok}[1]{\textcolor[rgb]{0.25,0.63,0.44}{#1}}
\newcommand{\BuiltInTok}[1]{\textcolor[rgb]{0.00,0.50,0.00}{#1}}
\newcommand{\CharTok}[1]{\textcolor[rgb]{0.25,0.44,0.63}{#1}}
\newcommand{\CommentTok}[1]{\textcolor[rgb]{0.38,0.63,0.69}{\textit{#1}}}
\newcommand{\CommentVarTok}[1]{\textcolor[rgb]{0.38,0.63,0.69}{\textbf{\textit{#1}}}}
\newcommand{\ConstantTok}[1]{\textcolor[rgb]{0.53,0.00,0.00}{#1}}
\newcommand{\ControlFlowTok}[1]{\textcolor[rgb]{0.00,0.44,0.13}{\textbf{#1}}}
\newcommand{\DataTypeTok}[1]{\textcolor[rgb]{0.56,0.13,0.00}{#1}}
\newcommand{\DecValTok}[1]{\textcolor[rgb]{0.25,0.63,0.44}{#1}}
\newcommand{\DocumentationTok}[1]{\textcolor[rgb]{0.73,0.13,0.13}{\textit{#1}}}
\newcommand{\ErrorTok}[1]{\textcolor[rgb]{1.00,0.00,0.00}{\textbf{#1}}}
\newcommand{\ExtensionTok}[1]{#1}
\newcommand{\FloatTok}[1]{\textcolor[rgb]{0.25,0.63,0.44}{#1}}
\newcommand{\FunctionTok}[1]{\textcolor[rgb]{0.02,0.16,0.49}{#1}}
\newcommand{\ImportTok}[1]{\textcolor[rgb]{0.00,0.50,0.00}{\textbf{#1}}}
\newcommand{\InformationTok}[1]{\textcolor[rgb]{0.38,0.63,0.69}{\textbf{\textit{#1}}}}
\newcommand{\KeywordTok}[1]{\textcolor[rgb]{0.00,0.44,0.13}{\textbf{#1}}}
\newcommand{\NormalTok}[1]{#1}
\newcommand{\OperatorTok}[1]{\textcolor[rgb]{0.40,0.40,0.40}{#1}}
\newcommand{\OtherTok}[1]{\textcolor[rgb]{0.00,0.44,0.13}{#1}}
\newcommand{\PreprocessorTok}[1]{\textcolor[rgb]{0.74,0.48,0.00}{#1}}
\newcommand{\RegionMarkerTok}[1]{#1}
\newcommand{\SpecialCharTok}[1]{\textcolor[rgb]{0.25,0.44,0.63}{#1}}
\newcommand{\SpecialStringTok}[1]{\textcolor[rgb]{0.73,0.40,0.53}{#1}}
\newcommand{\StringTok}[1]{\textcolor[rgb]{0.25,0.44,0.63}{#1}}
\newcommand{\VariableTok}[1]{\textcolor[rgb]{0.10,0.09,0.49}{#1}}
\newcommand{\VerbatimStringTok}[1]{\textcolor[rgb]{0.25,0.44,0.63}{#1}}
\newcommand{\WarningTok}[1]{\textcolor[rgb]{0.38,0.63,0.69}{\textbf{\textit{#1}}}}
\setlength{\emergencystretch}{3em} % prevent overfull lines
\providecommand{\tightlist}{%
  \setlength{\itemsep}{0pt}\setlength{\parskip}{0pt}}
\setcounter{secnumdepth}{-\maxdimen} % remove section numbering
\ifLuaTeX
  \usepackage{selnolig}  % disable illegal ligatures
\fi
\IfFileExists{bookmark.sty}{\usepackage{bookmark}}{\usepackage{hyperref}}
\IfFileExists{xurl.sty}{\usepackage{xurl}}{} % add URL line breaks if available
\urlstyle{same} % disable monospaced font for URLs
\hypersetup{
  hidelinks,
  pdfcreator={LaTeX via pandoc}}

\author{}
\date{}

\usepackage[margin=1in]{geometry}
\begin{document}


\hypertarget{procsyskernelpid_max}{%
\paragraph{/proc/sys/kernel/pid\_max}\label{procsyskernelpid_max}}

\begin{verbatim}
defines max pid in linux systems
\end{verbatim}

\hypertarget{the-parent-of-any-process-can-be-found-by-looking-at-the-ppid-field-provided-in-the-linux-specific-procpidstatus-file}{%
\subsection{The parent of any process can be found by looking at the
Ppid field provided in the Linux-specific /proc/PID/status
file}\label{the-parent-of-any-process-can-be-found-by-looking-at-the-ppid-field-provided-in-the-linux-specific-procpidstatus-file}}

\begin{itemize}
\tightlist
\item
  Although not specified in SUSv3, the C program environment on most
  UNIX implementations (including Linux) provides three global symbols:
  etext, edata, and end. These symbols can be used from within a program
  to obtain the addresses of the next byte past, respectively, the end
  of the program text, the end of the initial- ized data segment, and
  the end of the uninitialized data segment. To make use of these
  symbols, we must explicitly declare them, as follows:
\end{itemize}

\begin{Shaded}
\begin{Highlighting}[]
\KeywordTok{extern} \DataTypeTok{char}\NormalTok{ etext}\OperatorTok{,}\NormalTok{ edata}\OperatorTok{,}\NormalTok{ end}\OperatorTok{;}
        \CommentTok{/* For example, \&etext gives the address of the end}
\CommentTok{           of the program text / start of initialized data */}
\end{Highlighting}
\end{Shaded}

\hypertarget{locality-of-reference}{%
\subsection{Locality of Reference}\label{locality-of-reference}}

Locality of reference refers to the tendency of programs to access
certain memory locations more frequently than others. This concept is
essential for understanding memory optimization techniques in computer
systems.

\hypertarget{types-of-locality}{%
\subsection{Types of Locality}\label{types-of-locality}}

\begin{enumerate}
\def\labelenumi{\arabic{enumi}.}
\tightlist
\item
  \textbf{Spatial Locality}\\
  Spatial locality is the tendency of a program to reference memory
  addresses that are near those recently accessed. This behavior often
  occurs due to:

  \begin{itemize}
  \tightlist
  \item
    Sequential processing of instructions.
  \item
    Sequential processing of data structures.
  \end{itemize}
\item
  \textbf{Temporal Locality}\\
  Temporal locality is the tendency of a program to access the same
  memory \#\# Resident Set the pages of a program need to be resident in
  physical memory page frames; these pages form the so-called
  \texttt{Resident\ Set}
\end{enumerate}

\begin{itemize}
\tightlist
\item
  Page Size in Unix :
\end{itemize}

\begin{Shaded}
\begin{Highlighting}[]
\DataTypeTok{long}\NormalTok{ x }\OperatorTok{=}\NormalTok{ sysconf}\OperatorTok{(}\NormalTok{\_SC\_PAGESIZE}\OperatorTok{)}
\end{Highlighting}
\end{Shaded}

\begin{itemize}
\tightlist
\item
  If a process tries to access an address for which there is no
  corresponding page-table entry, it receives a \textbf{SIGSEGV} signal
\item
  Where appropriate, two or more processes can share memory. The kernel
  makes this possible by having page-table entries in different
  processes refer to the same pages of RAM. Memory sharing occurs in two
  common circumstances: -- Multiple processes executing the same program
  can share a single (read- only) copy of the program code. This type of
  sharing is performed implicitly when multiple programs execute the
  same program file (or load the same shared library). -- Processes can
  use the shmget() and mmap() system calls to explicitly request sharing
  of memory regions with other processes. This is done for the pur- pose
  of interprocess communication.
\end{itemize}

\hypertarget{user-stack-vs-kernel-stack}{%
\subsection{User stack vs Kernel
Stack}\label{user-stack-vs-kernel-stack}}

Sometimes, the term user stack is used to distinguish the stack we
describe here from the kernel stack. The kernel stack is a per-process
memory region maintained in kernel memory that is used as the stack for
execution of the functions called internally during the execution of a
system call. (The kernel can't employ the user stack for this purpose
since it resides in unprotected user memory.)

Each (user) stack frame contains the following information: -
\textbf{Function arguments and local variables:} In C these are referred
to as automatic variables, since they are automatically created when a
function is called. These variables also automatically disappear when
the function returns (since the stack frame disappears), and this forms
the primary semantic distinction between automatic and static (and
global) variables: the latter have a perma- nent existence independent
of the execution of functions. - \textbf{Call linkage information:} Each
function uses certain CPU registers, such as the program counter, which
points to the next machine-language instruction to be executed. Each
time one function calls another, a copy of these registers is saved in
the called function's stack frame so that when the function returns, the
appropriate register values can be restored for the calling function.

\hypertarget{why-argv0}{%
\subsection{Why argv{[}0{]} ?}\label{why-argv0}}

The fact that argv{[}0{]} contains the name used to invoke the program
can be employed to perform a useful trick. We can create multiple links
to (i.e., names for) the same program, and then have the program look at
argv{[}0{]} and take different actions depending on the name used to
invoke it. An example of this technique is provided by the gzip(1),
gunzip(1), and zcat(1) commands, all of which are links to the same
executable file. (If we employ this technique, we must be careful to
handle the possibility that the user might invoke the program via a link
with a name other than any of those that we expect.)

\end{document}
